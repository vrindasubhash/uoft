\documentclass[11pt]{article}
\usepackage{amsmath}
\usepackage{amsthm}
\usepackage{amsfonts}
\usepackage[utf8]{inputenc}
\usepackage[margin=0.75in]{geometry}

\title{CSC110 Fall 2022 Assignment 2: Logic, Constraints, and Wordle!}
\author{Vrinda Subhash}
\date{\today}

\begin{document}
\maketitle

\section*{Part 1: Conditional Execution}

Complete this part in the provided \texttt{a2\_part1\_q1\_q2.py} and \texttt{a2\_part1\_q3.py} starter files.
Do \textbf{not} include your solutions in this file.

\section*{Part 2: Proof and Algorithms, Greatest Common Divisor edition}

\begin{enumerate}
\item[1.]

This approach uses range(1, m + 1) instead of range(1, n + 1) because m is less than (or equal to) n. One of the properties of divisibility we learned in Lecture 9, is that for all n and d that are positive integers, if d divides n, then d is less than or equal to n. Using this property, we know that the greatest common divisor must be less than or equal to m and n. Since n might be greater than m, range(1, n + 1) might give us a list of invalid possible\_divisors (as they might be greater than m which goes against the divisibility property).

\item[2.]

We know the common\_divisors will not be an empty collection because possible\_divisors includes 1 in the range(1,  m + 1). Using the divisibility property from Lecture 9 that states that every integer is divisible by 1, we know that common\_divisors will at least contain the value 1, therefore it cannot be an empty collection, and the max function can be called on it.

\item[3.]

\begin{proof}
Want to show: $\forall n, m, d \in \mathbb{Z}, d|m \land m \neq 0 \implies (d|n \iff d|n \% m)$\\
Let n, m, d $\in \mathbb{Z}$\\
I can assume: $d|m \land m\neq 0$ \\
I want to show that $d|n \iff d|n \% m $ \\
To show that, I need to prove:\\
(1) $d|n \implies d|n \%m$ \\
(2) $d|n \%m \implies d|n$ \\
Proving (1):\\
Assume $d|n$. \\
We known $d|m$ since it was one of our assumptions. \\
Want to show: $d|n \%m$. \\
(a) Since $d|n, \exists a \in \mathbb{Z}$ such that n = da \\
(b) Since $d|m, \exists b \in \mathbb{Z}$ such that m = db \\
d is the gcd. \\
By Quotient Remainder Theorem: \\
$0 \leq r < |m|$ and $\exists c \in \mathbb{Z}$ such that \\
$n \% m = r \rightarrow mc + r = n$ \\
Substituting values from (a) and (b) for n and m: \\
$(db)c + r = da$ \\
$dbc + r = da$ \\
$r = da - dbc$ \\
$r = d(a-bc)$ \\
$(a-bc) \in \mathbb{Z}$ because $a, b, c \in \mathbb{Z}$ \\
This shows that $d|r$ because $\exists p \in \mathbb{Z}$ such that r = dp. \\
$r = n \% m$ \\
Therefore: $d|n\%m$ \\
Proving (2): \\
Assume $d|n\%m$. \\
We know $d|m$ since it was one of our assumptions. \\
Want to show: $d|n$ \\
Since $d|m, \exists b \in \mathbb{Z}$ such that $m=bd$. \\
This means m is a multiple of d. \\
Since $d|n\%m, n\%m$ is a multiple of d. \\
By Quotient Remainder Theorem: \\
$0 \leq r < |m|$ and $\exists c \in \mathbb{Z}$ such that \\
$n \% m = r \rightarrow mc + r = n$ \\
mc is a multiple of d because $m = bd$, and $c \in \mathbb{Z}$. \\
r is a multiple of d because $r = n \% m$ and $d|n\%m$. \\
$\exists x \in \mathbb{Z}$ such that $mc = dx$ \\
$\exists y \in \mathbb{Z}$ such that $r = dy$ \\
$mc + r = n$ can be rewritten as: \\
$dx + dy = n$ \\
$d(x+y) = n$ \\
This shows that n is a multiple of d. \\
Therefore: $d|n$ \\
I have now proven (1) and (2) which proves that $d|n \iff d|n \% m $. \\
Therefore: $\forall n, m, d \in \mathbb{Z}, d|m \land m \neq 0 \implies (d|n \iff d|n \% m)$.
\end{proof}

\item[4.]

If the mod of two numbers is 0, the gcd is the value you divided by, which is why I added "return m" after the first branch of the if-statement. The other change I made is changing the range of possible\_divisors to be range(1, r + 1) because the remainder will be a multiple of the gcd, and the remainder will be greater than or equal to the gcd, so when looking for the gcd, you know it will be in the range(1, r+1). And r is less than m which is what makes the range(1, r+1), smaller than range(1, m+1).

\begin{verbatim}
def gcd(n: int, m: int) -> int:
    """Return the greatest common divisor of m and n.

    Preconditions:
    - 1 <= m <= n
    """
    r = n % m

    if r == 0:
        return m
    else:
        possible_divisors = range(1, r + 1)
        common_divisors = {d for d in possible_divisors if divides(d, n) and divides(d, m)}
        return max(common_divisors)
\end{verbatim}
\end{enumerate}


\section*{Part 3: Wordle!}

Complete this part in the provided \texttt{a2\_part3.py} starter file.
Do \textbf{not} include your solutions in this file.

\end{document}
